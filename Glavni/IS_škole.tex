\documentclass{article}
\usepackage[utf8]{inputenc}
\usepackage{enumitem}
\usepackage{cmsrb}
\usepackage{graphicx}
\usepackage[OT2,T1]{fontenc} 
\usepackage[serbian]{babel}


\begin{document}

\begin{titlepage}
    \centering

    \newcommand{\HRule}{\rule{\linewidth}{0.5mm}}
    \center
    \textup{\Large Univerzitet u Beogradu\\Matematički fakultet}\\[1.5cm]
    \textup{\Large Projekat iz predmeta Informacioni sistemi 2022/2023.}\\[0.4cm]

    \HRule \\[0.4cm]
    { \huge \bfseries Informacioni sistem srednje škole}\\[0.4cm]
    \HRule \\[1.1cm]
    
    \vspace{6 cm}
	
	\begin{minipage}{0.4\textwidth}
		\begin{flushleft} \large
			\emph{Autori:}\\
			Pavle Savić\\
			Jovan Marković\\
			Mateja Trtica\\
			Lazar Ristić\\		
		\end{flushleft}
	\end{minipage}
	~
	\begin{minipage}{0.5\textwidth}
		\begin{flushright} \large
			\emph{Predmet:}\\ 
			Informacioni sistemi\\
			\emph{Profesor:}\\ 
			Saša Malkov\\
			\emph{Asistent:}\\ 
			Dara Milojković
		\end{flushright}
	\end{minipage}\\[2cm]
	{\large \today}\\[2cm]
\end{titlepage}

\thispagestyle{empty}

\newpage
\renewcommand*\contentsname{Sadržaj:}
\tableofcontents


\newpage
\section{Analiza sistema}

\subsection{Uvod i osnovna ideja}
\subsection{Korisnici sistema}
\subsection{Kratak opis}


\newpage
\section{Dijagrami toka podataka}

\begin{figure} [!ht]
    \begin{center}
        \includegraphics[scale=0.5]{imgs/dijagram_konteksta.png}
    \end{center}
\caption{Dijagram konteksta sistema}
\end{figure}

TODO: dijagram nivoa 0

\newpage
\section{Slučajevi upotrebe}

\subsection{Aktivnosti Učenika}

\begin{figure} [!ht]
    \begin{center}
        \includegraphics[scale=0.6]{imgs/ucenik_use_case.png}
    \end{center}
\caption{Dijagram slučajeva upotrebe Učenika}
\end{figure}


\newpage
\subsubsection{Slučaj upotrebe: Učenik pristupa svojim ocenama} 
1. \textbf{Kratak opis:} Učenik pristupa svojim ocenama iz željenog predmeta. Učenik filtrira ocene po željenom kriterijumu. \\

2. \textbf{Učesnici:}
\begin{enumerate} [label=(\alph*)]
\item Učenik
\end{enumerate} 

3. \textbf{Preduslov:} Učenik je registrovani korisnik sistema. Učenik ima pristup Internetu. Sistem je u funkciji. Učenik je upisan na predmet. \\

4. \textbf{Postuslov:} Učenik je pristupio elektronskom dnevniku i pročitao potrebne informacije o ocenama.\\

5. \textbf{Osnovni tok:} 
\begin{enumerate} [label=(\alph*)]
\item Učenik pristupa stranici sa listom predmeta koje pohađa
\item Učenik pritiska dugme za pristup ocenama iz željenog predmeta
\item Sistem prikazuje sve ocene učenika iz tog predmeta u tekućoj školskoj godini
\item Učenik pritiska na dugme za filtriranje ocena
\item Sistem prikazuje učeniku polja za filtriranje
\item Učenik unosi željeni kriterijum filtriranja
\item Sistem filtrira ocene
\item Sistem prikazuje ocene filtrirane po željenom kriterijumu
\end{enumerate}

6. \textbf{Alternativni tokovi:}
\begin{enumerate} [label=(\roman*)]
\item Učenik nema ocene za prikaz - Ako u koraku (b) učenik želi da pristupi ocenama iz predmeta iz kojeg nije upisana nijedna ocena sistem ga obaveštava odgovarajućom porukom. Proces se nastavlja iz koraka (a).
\item Nijedna ocena ne ispunjava kriterijum - Ako u koraku (g) sistem utvrdi da nijedna ocena ne ispunjava željeni kriterijum obaveštava učenika odgovarajućom porukom. Proces se nastavlja iz koraka (c).
\end{enumerate}

7. \textbf{Podtokovi}: - \\

8. \textbf{Specijalni zahtevi}: - \\

9. \textbf{Dodatne informacije}: Klikom na proizvoljnu ocenu učenik može videti tip ocene (kontrolni/pismeni/usmeni/aktivnost) i napomenu (ako je uneta). Kriterijum filtriranja su tip ocene i polugodište (prvo ili drugo). \\

\subsubsection{Slučaj upotrebe: Učenik pristupa rasporedu časova} 

1. \textbf{Kratak opis:} Učenik pristupa rasporedu časova odeljenja kom pripada u željenoj smeni. \\

2. \textbf{Učesnici:}
\begin{enumerate} [label=(\alph*)]
\item Učenik
\end{enumerate} 

3. \textbf{Preduslov:} Učenik je registrovani korisnik sistema. Učenik ima pristup Internetu. Sistem je u funkciji. Učenik je dodeljen odeljenju. \\

4. \textbf{Postuslov:} Učenik je pristupio rasporedu časova i pročitao potrebne informacije. \\

5. \textbf{Osnovni tok:} 
\begin{enumerate} [label=(\alph*)]
\item Učenik pristupa početnoj stranici rasporeda časova
\item Sistem prikazuje učeniku padajući meni za izbor odeljenja i smene
\item Učenik pritiskom na željene opcije bira odeljenje i smenu
\item Učenik pritiskom na dugme potvrđuje unos 
\item Sistem obrađuje zahtev
\item Sistem prikazuje učeniku raspored časova njegovog odeljenja u željenoj smeni 
\end{enumerate}

6. \textbf{Alternativni tokovi:}
\begin{enumerate} [label=(\roman*)]
\item Raspored časova ne postoji - Ako u koraku (e) sistem utvrdi da odeljenju nije dodeljen raspored časova (za odabranu smenu) obaveštava učenika odgovarajućom porukom. Proces se nastavlja iz koraka (b).

\end{enumerate}

7. \textbf{Podtokovi}: - \\

8. \textbf{Specijalni zahtevi}: - \\

9. \textbf{Dodatne informacije}: Prema pravilniku škole za svako odeljenje prepodnevna i poslepodnevna smena smenjuju se na nedeljnom nivou. \\


\subsubsection{Slučaj upotrebe: Učenik pristupa rasporedu redara} 
1. \textbf{Kratak opis:} Učenik pristupa rasporedu redara odeljenja kom pripada za željenu nedelju. \\ 

2. \textbf{Učesnici:}
\begin{enumerate} [label=(\alph*)]
\item Učenik
\end{enumerate} 

3. \textbf{Preduslov:} Učenik je registrovani korisnik sistema. Učenik ima pristup Internetu. Sistem je u funkciji. Učenik je dodeljen odeljenju. \\

4. \textbf{Postuslov:} Učenik je pristupio rasporedu redara i pročitao potrebne informacije. \\

5. \textbf{Osnovni tok:} 
\begin{enumerate} [label=(\alph*)]
\item Učenik pristupa početnoj stranici sa rasporedom redara svog odeljenja
\item Učenik se pritiskom na dugme (``Levo''/``Desno'') kreće do rasporeda redara za traženu nedelju
\item Sistem prikazuje raspored redara za traženu nedelju
\end{enumerate}


6. \textbf{Alternativni tokovi:}
\begin{enumerate} [label=(\roman*)]
\item Raspored redara za traženu nedelju ne postoji - Ako u koraku (c) sistem utvrdi da raspored redara za traženu nedelju nije određen, obaveštava učenika odgovarajućom porukom. Proces se nastavlja iz koraka (b).
\end{enumerate}

7. \textbf{Podtokovi}: - \\

8. \textbf{Specijalni zahtevi}: - \\

9. \textbf{Dodatne informacije}:  U toku nedelje dužnost redara u odeljenju obavlja dvoje učenika tog odeljenja. \\


\subsubsection{Slučaj upotrebe: Učenik se prijavljuje na vannastavnu aktivnost} 
1. \textbf{Kratak opis:} Učenik odabira željenu vannastavnu aktivnost. Sistem proverava raspoloživost, čuva promenu i o tome obaveštava učenika. \\

2. \textbf{Učesnici:}
\begin{enumerate} [label=(\alph*)]
\item Učenik
\end{enumerate} 

3. \textbf{Preduslov:} Učenik je registrovani korisnik sistema. Učenik ima pristup Internetu. Sistem je u funkciji. Raspored vannastavnih aktivnosti je kreiran. \\

4. \textbf{Postuslov:} Sistem je evidentirao prijavu učenika na vannastavnu aktivnost. Baza je ažurirana. \\

5. \textbf{Osnovni tok:} 
\begin{enumerate} [label=(\alph*)]
\item Učenik pristupa stranici sa rasporedom vannastavnih aktivnosti
\item Učenik pritiska na željenu aktivnost
\item Sistem prikazuje kratak opis odabrane aktivnosti
\item Učenik pritiska dugme za prijavu na željenu aktivnost
\item Sistem prikazuje dialog box (da/ne) i pita učenika da li želi da se prijavi
\item Učenik potvrđuje prijavu
\item Sistem vrši proveru raspoloživosti odabrane aktivnosti
\item Sistem čuva prijavu
\item Sistem obaveštava učenika o uspešnoj prijavi na vannastavnu aktivnost
\end{enumerate}

6. \textbf{Alternativni tokovi:}
\begin{enumerate} [label=(\roman*)]
\item Učenik ne potvrđuje prijavu - Ako u koraku (f) učenik odgovori odrično ili ugasi dialog box sistem ne evidentira prijavu. Proces se nastavlja iz koraka (c). 
\item Sva mesta na izabranoj aktivnosti popunjena - Ako u koraku (g) sistem utvrdi da učenik pokušava da se prijavi na aktivnost na kojoj su sva mesta popunjena, sistem o tome obaveštava učenika i odbija prijavu. Proces se nastavlja iz koraka (a).
\item Učenik se prijavljuje na već prijavljenu aktivnost - Ako u koraku (g) sistem utvrdi da učenik pokušava ponovno da se prijavi na aktivnost na koju je već prijavljen, sistem o tome obaveštava učenika i odbija prijavu. Proces se nastavlja iz koraka (a).

\end{enumerate}

7. \textbf{Podtokovi}: - \\

8. \textbf{Specijalni zahtevi}: Za željenu aktivnost dodeljen je koordinator. \\

9. \textbf{Dodatne informacije}: Škola nudi vannastavne aktivnosti kao što su: sportske sekcije, glumačka sekcija, muzička sekcija, besedništvo. Broj raspoloživih mesta specifičan je za svaku sekciju. \\

\newpage
\subsection{Aktivnosti Nastavnika}

\begin{figure} [!ht]
    \begin{center}
        \includegraphics[scale=0.45]{imgs/nastavnik_use_case.png}
    \end{center}
\caption{Dijagram slučajeva upotrebe Nastavnika i Razrednog starešine kao njegove specijalizacije}
\end{figure}

\newpage
\subsubsection{Slučaj upotrebe: Nastavnik evidentira održani čas} 
1. \textbf{Kratak opis:} Nakon održanog časa nastavnik unosi informacije o nastavnoj temi pređenoj na tom času i upisuje odsutne učenike. Sistem validira i čuva unete podatke i obaveštava nastavnika o uspešno unetom izveštaju.\\

2. \textbf{Učesnici:}
\begin{enumerate} [label=(\alph*)]
\item Nastavnik
\end{enumerate} 

3. \textbf{Preduslov:} Nastavnik je registrovani korisnik sistema. Nastavnik ima pristup Internetu. Sistem je u funkciji. \\

4. \textbf{Postuslov:} Izveštaj sa održanog časa je unet u arhivu dokumenata sistema. Odsutnim učenicima zabeležen je izostanak. Baza je ažurirana. \\

5. \textbf{Osnovni tok:} 
\begin{enumerate} [label=(\alph*)]
\item Nastavnik pristupa listi održanih časova
\item Nastavnik pritiska dugme za kreiranje novog izveštaja sa časa
\item Sistem prikazuje formular za kreiranje novog izveštaja sa časa
\item Nastavnik unosi tražene podatke 
\item Nastavnik pritiskom na dugme potvrđuje unos
\item Sistem vrši validaciju podataka
\item Sistem čuva podatke
\item Sistem vraća poruku o uspešno kreiranom izveštaju
\end{enumerate}

6. \textbf{Alternativni tokovi:}
\begin{enumerate} [label=(\roman*)]
\item Nastavnik unosi nevalidne podatke u formular - Ako u koraku (f) sistem utvrdi neispravno polje formulara sistem obaveštava nastavnika obeležavanjem nevalidnog polja. Nastavnik ispravlja neispravno polje. Proces se nastavlja iz koraka (d).
\end{enumerate}

7. \textbf{Podtokovi}: - \\

8. \textbf{Specijalni zahtevi}: - \\

9. \textbf{Dodatne informacije}: Obavezna polja formulara jesu: datum, oznaka odeljenja, nastavna tema i spisak odsutnih učenika. Opciona polja su: literatura za temu, domaći zadatak i napomene. \\


\subsubsection{Slučaj upotrebe: Nastavnik unosi dopunske i dodatne časove u raspored časova } 
1. \textbf{Kratak opis:} U dogovoru sa zainteresovanim učenicima nastavnik dodaje dodatni ili dopunski čas u postojeći raspored časova. Sistem validira i čuva unete izmene i obaveštava nastavnika o uspešno ažuriranom rasporedu. \\

2. \textbf{Učesnici:}
\begin{enumerate} [label=(\alph*)]
\item Nastavnik
\end{enumerate} 

3. \textbf{Preduslov:} Nastavnik je registrovani korisnik sistema. Nastavnik ima pristup Internetu. Sistem je u funkciji. \\

4. \textbf{Postuslov:} Dopunski ili dodatni čas dodat je u raspored časova. Baza je ažurirana. \\

5. \textbf{Osnovni tok:} 
\begin{enumerate} [label=(\alph*)]
\item Nastavnik pristupa stranici na kojoj je prikazan raspored časova
\item Nastavnik pritiska na dugme ispod odgovarajućeg dana  
\item Sistem otvara formular za upis podataka o času koji se dodaje u raspored
\item Nastavnik popunjava formular traženim podacima
\item Nastavnik pritiskom na dugme potvrđuje slanje zahteva za unos časa u raspored
\item Sistem validira unete podatke
\item Sistem vrši proveru da li je došlo do preklapanja u rasporedu časova
\item Sistem čuva podatke
\item Sistem vraća poruku o uspešno ažuriranom rasporedu
\end{enumerate}

6. \textbf{Alternativni tokovi:}
\begin{enumerate} [label=(\roman*)]
\item Nastavnik unosi nevalidne podatke u formular - Ako u koraku (f) sistem utvrdi nevalidne podatke (obavezna polja ostala nepopunjena) obaveštava nastavnika obeležavanjem neispravnog polja. Nastavnik ispravlja neispravno polje. Proces se nastavlja iz koraka (d).
\item Neuspešna izmena rasporeda - Ako u koraku (g) sistem utvrdi preklapanje u rasporedu časova obaveštava nastavnika ispisivanjem časa koji se u rasporedu već nalazi u tom terminu. Nastavnik bira drugi dan i proces se nastavlja iz koraka (b).

\end{enumerate}

7. \textbf{Podtokovi}: - \\

8. \textbf{Specijalni zahtevi}: Nastavnik dodatnu/dopunsku nastavu može zakazati samo u terminu 7. časa u prvoj smeni (predčas u drugoj smeni). \\

9. \textbf{Dodatne informacije}: Obavezna polja formulara su: naziv predmeta, odeljenje, tip časa (dopunska/dodatna). Opciono polje je napomena. \\


\subsubsection{Slučaj upotrebe: Nastavnik upisuje ocenu učeniku} 
1. \textbf{Kratak opis:} Nakon održane provere znanja (pismene ili usmene) nastavnik unosi ostvarenu ocenu učenika u elektronski dnevnik. Sistem validira i čuva unetu ocenu i obaveštava nastavnika o uspešno unetoj oceni.  \\

2. \textbf{Učesnici:}
\begin{enumerate} [label=(\alph*)]
\item Nastavnik
\end{enumerate} 

3. \textbf{Preduslov:} Nastavnik je registrovani korisnik sistema. Nastavnik ima pristup Internetu. Sistem je u funkciji. \\

4. \textbf{Postuslov:} Ocena učenika uneta je u elektronski dnevnik. Baza je ažurirana.\\

5. \textbf{Osnovni tok:} 
\begin{enumerate} [label=(\alph*)]
\item Nastavnik pristupa listi učenika podeljenih po odeljenjima kojima predaje
\item Nastavnik vrši pretragu i pritiska dugme za pristup ocenama kod odgovarajućeg učenika
\item Sistem otvara stranicu sa ocenama odgovarajućeg učenika
\item Nastavnik pritiska na dugme za unos nove ocene
\item Sistem prikazuje obrazac za unos nove ocene
\item Nastavnik popunjava obrazac
\item Nastavnik pritiskom na dugme potvrđuje unos ocene
\item Sistem validira unete podatke
\item Sistem čuva unetu ocenu
\item Sistem vraća poruku o uspešno unetoj oceni
\end{enumerate}

6. \textbf{Alternativni tokovi:}
\begin{enumerate} [label=(\roman*)]
\item Nastavnik unosi nevalidne podatke u obrazac - Ako u koraku (h) sistem utvrdi nevalidnu ocenu (ocena nije od 1 do 5) ili prazno obavezno polje obaveštava nastavnika obeležavanjem neispravnog polja. Nastavnik ispravlja neispravno polje. Proces se nastavlja iz koraka (f).
\end{enumerate}

7. \textbf{Podtokovi}: - \\

8. \textbf{Specijalni zahtevi}: - \\

9. \textbf{Dodatne informacije}: Obavezna polja obrasca su: ocena i tip ocene (kontrolni/pismeni/usmeni/aktivnost). Opciono polje je napomena. \\


\subsubsection{Slučaj upotrebe: Nastavnik zaključuje ocenu učeniku} 
1. \textbf{Kratak opis:} Na osnovu ostvarenih ocena u toku školske godine nastavnik unosi zaključnu ocenu učenika u elektronski dnevnik. Sistem validira i čuva unetu ocenu i obaveštava nastavnika o uspešnom zaključivanju ocene. \\

2. \textbf{Učesnici:}
\begin{enumerate} [label=(\alph*)]
\item Nastavnik
\end{enumerate} 

3. \textbf{Preduslov:} Nastavnik je registrovani korisnik sistema. Nastavnik ima pristup Internetu. Sistem je u funkciji. \\

4. \textbf{Postuslov:} Zaključna ocena učenika uneta je u elektronski dnevnik. Baza je ažurirana. \\

5. \textbf{Osnovni tok:} 
\begin{enumerate} [label=(\alph*)]
\item Nastavnik pristupa listi učenika podeljenih po odeljenjima kojima predaje
\item Nastavnik vrši pretragu i pritiska dugme za pristup ocenama kod odgovarajućeg učenika
\item Sistem otvara stranicu sa ocenama odgovarajućeg učenika
\item Nastavnik pritiska na dugme za otvaranje obrasca za zaključivanje ocene
\item Sistem proverava uslove za zaključivanje
\item Sistem prikazuje nastavniku obrazac za zaključivanje 
\item Nastavnik popunjava obrazac
\item Nastavnik pritiskom na dugme potvrđuje zaključivanje ocene
\item Sistem validira unetu zaključnu ocenu
\item Sistem čuva unetu zaključnu ocenu
\item Sistem vraća poruku o uspešno zaključenoj oceni
\end{enumerate}


6. \textbf{Alternativni tokovi:}
\begin{enumerate} [label=(\roman*)]
\item Nastavnik pokuša nedozvoljeno zaključivanje ocene - Ako u koraku (e) sistem utvrdi da nisu ispunjeni uslovi za zaključivanje ocene (učenik nema dovoljno ocena) obaveštava nastavnika da ocenu nije moguće zaključiti. Proces se završava.
\item Nastavnik zaključuje nevalidnu ocenu - Ako u koraku (i) sistem utvrdi nevalidnu zaključnu ocenu (ocena nije od 1 do 5, prazno polje, ocena je manja od aritmetičke sredine ocena učenika) obaveštava nastavnika obeležavanjem neispravnog polja obrasca. Nastavnik ispravlja ocenu. Proces se nastavlja iz koraka (g).
\end{enumerate}

7. \textbf{Podtokovi}: - \\

8. \textbf{Specijalni zahtevi}: Ocenu moguće zaključiti samo u poslednjoj radnoj nedelji školske godine. \\

9. \textbf{Dodatne informacije}: Minimalan broj ocena učenika da bi ocena mogla da se zaključi zavisi od fonda časova odgovarajućeg predmeta i definisan je pravilnikom škole. Obavezno polje obrasca je zaključna ocena. Opciono polje je napomena. \\

\newpage
\subsection{Aktivnosti Razrednog starešine}

\subsubsection{Slučaj upotrebe: Razredni starešina opravdava učeniku evidentirani izostanak}
1. \textbf{Kratak opis:} Na osnovu opravdanja koje je učenik dostavio uživo ili elektronski skenirano kroz sistem razredni starešina evidentira izostanak kao opravdan. Sistem čuva promenu statusa odgovarajućeg izostanka i o tome obaveštava razrednog starešinu. \\

2. \textbf{Učesnici:}
\begin{enumerate} [label=(\alph*)]
\item Razredni starešina
\item Učenik
\end{enumerate} 

3. \textbf{Preduslov:} Razredni starešina je registrovani korisnik sistema. Razredni starešina ima pristup Internetu. Sistem je u funkciji. Učenik je dostavio opravdanje. \\

4. \textbf{Postuslov:} Sistem je evidentirao izostanak učenika kao opravdan. Baza je ažurirana. \\

5. \textbf{Osnovni tok:} 
\begin{enumerate} [label=(\alph*)]
\item Razredni starešina pristupa stranici sa listom učenika u svom odeljenju
\item Razredni starešina pritiska na dugme za pristup korisničkom profilu odgovarajućeg učenika
\item Sistem otvara profil učenika
\item Razredni starešina pritiska na dugme za izlistavanje izostanaka učenika
\item Sistem prikazuje listu izostanaka i status svakog izostanka (opravdan/neopravdan)
\item Razredni starešina pritiska odgovarajući neopravdani izostanak
\item Sistem prikazuje dialog box (da/ne) i pita razrednog starešinu da li želi da promeni status neopravdanog izostanka
\item Razredni starešina potvrđuje promenu
\item Sistem čuva promenu statusa izostanka
\item Sistem obaveštava razrednog starešinu o uspešnoj promeni statusa izostanka
\end{enumerate}

6. \textbf{Alternativni tokovi:}
\begin{enumerate} [label=(\roman*)]
\item Razredni starešina ne potvrđuje promenu statusa izostanka - Ako u koraku (h) razredni starešina odgovori odrično ili ugasi dialog box sistem ne evidentira promenu statusa. Proces se nastavlja iz koraka (e). 
\end{enumerate}

7. \textbf{Podtokovi}: - \\

8. \textbf{Specijalni zahtevi}: Izostanak se mora opravdati u roku od dve nedelje nakon evidentiranja. Nakon isteka roka razredni starešina više nema mogućnost opravdavanja i status izostanka ostaje - neopravdan. \\

9. \textbf{Dodatne informacije}: - \\

\begin{figure} [!ht]
    \begin{center}
        \includegraphics[scale=0.35]{imgs/BPMN_opravdavanje_casova.png}
    \end{center}
\caption{BPMN dijagram saradnje Opravdavanje izostanka učeniku}
\end{figure}

\subsubsection{Slučaj upotrebe: Razredni starešina izriče učeniku lakšu disciplinsku meru}
1. \textbf{Kratak opis:} Na osnovu donesene odluke razredni starešina unosi u sistem lakšu disciplinsku meru izrečenu učeniku. Sistem validira i čuva unete podatke i o tome obaveštava razrednog starešinu. \\

2. \textbf{Učesnici:}
\begin{enumerate} [label=(\alph*)]
\item Razredni starešina
\end{enumerate} 

3. \textbf{Preduslov:} Razredni starešina je registrovani korisnik sistema. Razredni starešina ima pristup Internetu. Sistem je u funkciji. \\

4. \textbf{Postuslov:} Dokument o izrečenoj meri je unet u arhivu dokumenata sistema. Disciplinska mera učenika evidentirana je u sistemu. Baza je ažurirana. \\

5. \textbf{Osnovni tok:} 
\begin{enumerate} [label=(\alph*)]
\item Razredni starešina pristupa stranici sa listom učenika u svom odeljenju
\item Razredni starešina pritiska na dugme za pristup korisničkom profilu odgovarajućeg učenika
\item Sistem otvara profil učenika
\item Razredni starešina pritiska dugme za kreiranje dokumenta o disciplinskoj meri
\item Sistem prikazuje obrazac za kreiranje dokumenta o disciplinskoj meri 
\item Razredni starešina unosi tražene podatke
\item Razredni starešina pritiskom na dugme potvrđuje unos
\item Sistem vrši validaciju podataka
\item Sistem čuva podatke
\item Sistem obaveštava razrednog starešinu o uspešno izrečenoj disciplinskoj meri
\end{enumerate}

6. \textbf{Alternativni tokovi:}
\begin{enumerate} [label=(\roman*)]
\item Razredni starešina unosi nevalidne podatke u obrazac - Ako u koraku (h) sistem utvrdi nevalidno polje obrasca (prazno obavezno polje) sistem obaveštava razrednog starešinu obeležavanjem neispravnog polja. Razredni starešina ispravlja neispravno polje. Proces se nastavlja iz koraka (f).
\end{enumerate}

7. \textbf{Podtokovi}: - \\

8. \textbf{Specijalni zahtevi}: U slučaju ukora odeljenskog veća prethodno je održana sednica na kojoj je disciplinska mere izglasana. \\

9. \textbf{Dodatne informacije}: Obavezna polja obrasca jesu: datum, nivo mere (opomena/ukor razrednog starešine/ukor odeljenskog veća), razlog izricanja mere. Opciono polje je napomena. Lakše disciplinske mere ne povlače smanjenje ocene iz vladanja. Odeljensko veće čine nastavnici koji izvode nastavu u određenom odeljenju. \\

\newpage
\subsection{Aktivnosti Direktora}

\begin{figure} [!ht]
    \begin{center}
        \includegraphics[scale=0.6]{imgs/direktor_use_case.png}
    \end{center}
\caption{Dijagram slučajeva upotrebe Direktora}
\end{figure}

\newpage
\subsubsection{Slučaj upotrebe: Direktor ažurira kalendar aktivnosti}
1. \textbf{Kratak opis:} Sistem direktoru prikazuje kalendar aktivnosti. Direktor može praviti izmene u kalendaru.\\

2. \textbf{Učesnici:}
\begin{enumerate} [label=(\alph*)]
\item Direktor
\end{enumerate} 

3. \textbf{Preduslov:} Direktor je registrovani korisnik sistema. Direktor ima pristup Internetu. Sistem je u funkciji.\\

4. \textbf{Postuslov:}  Kalendar aktivnosti je ažuriran novim informacijama.\\

5. \textbf{Osnovni tok:} 
\begin{enumerate} [label=(\alph*)]
\item Direktor klikće na dugme " Prikaži kalendar aktivnosti" 
\item Sistem prikazuje direktoru kalendar aktivnosti
\item Direktor klikom na dugme "Izmeni kalendar aktivnosi" pralazi na podtok slučaja upotrebe (1) 
\end{enumerate}

6. \textbf{Alternativni tokovi:}
\begin{enumerate} [label=(\roman*)]
\item Nedovoljan broj radnih dana - Ukoliko u koraku (6) u podtoku (1) nakon izmene broj radnih dana ne ispunjava zakonski minimum,  Sistem šalje obaveštenje direktoru i onemogućava se klik na dugme "Završi" i prelazi se na korak (1).
\end{enumerate}

7. \textbf{Podtokovi}:  \\
Podtok 1:
\begin{enumerate} [label=(\alph*)]
\item Sistem direktoru prikazuje novi formular sa opcijama " {Dodaj nastavni dan} ", " {Dodaj vannastavnu aktivnost} " i "{ Dodaj} nenastavni dan"
\item Direktor klikće na odgovarajuće dugme
\item Sistem prikazuje direktoru formu u kojoj određuje datumski interval važenja izmene
\item Direktor unosi datum u formular
\item Sistem unosi izmenu u kalendar aktivnosti
\item Sistem šalje poruku o uspešnoj izmeni kalendara aktivnosti
\item Direktor klikće na dugme " {Završi} " 
\item Sistem vraća direktora na korak (2) iz osnovnog toka 
\end{enumerate} 

8. \textbf{Specijalni zahtevi}: - \\

9. \textbf{Dodatne informacije}: - \\


\subsubsection{Slučaj upotrebe: Direktor ažurira dnevnike} 
1. \textbf{Kratak opis:}  Sistem prikazuje direktoru sve dnevnike. Direktor može vršiti izmene u dnevnicima. \\

2. \textbf{Učesnici:}
\begin{enumerate} [label=(\alph*)]
\item Direktor
\end{enumerate} 

3. \textbf{Preduslov:} Direktor je registrovani korisnik sistema. Direktor ima pristup Internetu. Sistem je u funkciji. \\

4. \textbf{Postuslov:}  Dnevnici su ažurirani ukoliko su izvršene izmene.\\

5. \textbf{Osnovni tok:} 
\begin{enumerate} [label=(\alph*)]
\item Direktor klikće na dugme " {Pogledaj dnevnike}" 
\item Sistem prikazuje sve postojeće dnevnike
\item Direktor odabira dnevnik željenog odeljenja
\item Sistem pokazuje direktoru informacije sadržane u dnevniku
\item Ukoliko direktor klikne na dugme " {Izmena} ocena"  {sistem} prebacuje izvršavanje na podtok slučaja upotrebe (1)
\item Ukoliko direktor klikne na dugme " {Dodeljivanje} razrednog starešine" sistem prebacuje izvršavanje na podtok slučaja upotrebe (2)
\item Ukoliko direktor klikne na dugme " {Pravdanje izostanka}" {sistem} prebacuje izvršavanje na podtok slučaja upotrebe (3)
\item Ukoliko direktor klikne na dugme " {Dodeli zamenu}"  {sistem} prebacuje izvršavanje na podtok slučaja upotrebe (4)
\item Direktor klikće na dugme "Nazad" 
\item Sistem vraća direktora na početni ekran

\end{enumerate}

6. \textbf{Alternativni tokovi:}
\begin{enumerate} [label=(\roman*)]
\item Učenik već postoji u sistemu - Ako sistem u koraku (4) utvrdi da učenik sa postojećim informacijama već postoji u sistemu, sistem ispisuje odgovarajuću poruku i vraća korisnika na korak (2) 
\item Dodeljeno odeljenje ne postoji - Ako sistem u koraku (9) utvrdi da dodeljeno odeljenje ne postoji, prikazuje odgovarajuću poruku i vraća korisnika na korak (8).
\end{enumerate}

7. \textbf{Podtokovi}:  \\
1: \\
\begin{enumerate} [label=(\alph*)]
\item Sistem prikazuje korisniku spisak učenika u odeljenju
\item Korisnik klikće na učenika
\item Sistem prikazuje sve ocene željenog učenika
\item Korisnik menja određenu ocenu
\item Korisnik klikće na  " {Završi} sa izmenama"
\item Sistem unosi izmene
\item Sistem vraća korisnika na korak (4) iz osnovnog toka
\end{enumerate}
2: \\
\begin{enumerate} [label=(\alph*)]
\item Sistem prikazuje korisniku spisak nastavnika
\item Korisnik klikće na nastavnika
\item Korisnik klikće na  " {Završi} sa izmenama"
\item Sistem unosi izmene
\item Sistem vraća korisnika na korak (4) iz osnovnog toka
\end{enumerate}
3: \\
\begin{enumerate} [label=(\alph*)]
\item Sistem prikazuje korisniku spisak učenika u odeljenju
\item Korisnik klikće na učenika
\item Sistem prikazuje sve izostanke željenog učenika
\item Korisnik menja određen izostanak
\item Korisnik klikće na  " {Završi} sa izmenama"
\item Sistem unosi izmene
\item Sistem vraća korisnika na korak (4) iz osnovnog toka
\end{enumerate}
4: \\
\begin{enumerate} [label=(\alph*)]
\item Sistem prikazuje korisniku spisak nastavnika koji predaju datom odeljenju
\item Korisnik klikće na nastavnika koji predaje tom odeljenju
\item Sistem prikazuje korisniku spisak nastavnika zaposlenih u školi
\item Korisnik klikće na nastavnika zaposlenog u školi
\item Sistem prikazuje korisniku sve predmete koje je nastavnik predavao odeljenju
\item Sistem prikazuje korisniku spisak nastavnika koji predaju datom odeljenju
\item Korisnik odabira predmete za koje želi da dodeli zamenu
\item Korisnik klikće na  " {Završi} sa izmenama"
\item Sistem unosi izmene
\item Sistem vraća korisnika na korak (4) iz osnovnog toka
\end{enumerate}

8. \textbf{Specijalni zahtevi}: - \\

9. \textbf{Dodatne informacije}: - \\


\subsubsection{Slučaj upotrebe: Direktor pravi finansijski izveštaj škole}
1. \textbf{Kratak opis:}  Direktor na osnovu finansijskih prihoda i rashoda pravi finansijski izveštaj škole. Sistem validira i čuva unete podatke i obaveštava direktora o uspešno unetom izveštaju. \\

2. \textbf{Učesnici:}
\begin{enumerate} [label=(\alph*)]
\item Direktor
\end{enumerate} 

3. \textbf{Preduslov:} Direktor je registrovani korisnik sistema. Direktor ima pristup Internetu. Sistem je u funkciji. \\

4. \textbf{Postuslov:} Finansijski izveštaj unet je u arhivu dokumenata sistema. \\

5. \textbf{Osnovni tok:} 
\begin{enumerate} [label=(\alph*)]
\item Direktor pristupa stranici sa finansijskim izveštajima
\item Direktor pritiska dugme za kreiranje novog izveštaja
\item Sistem kreira prazan finansijski izveštaj
\item Direktor pritiska dugme za dodavanje nove transakcije u izveštaj
\item Sistem prikazuje formular za unošenje informacija o novoj transakciji
\item Direktor unosi informacije o transakciji
\item Direktor pritiskom na dugme potvrđuje unos
\item Sistem vrši validaciju podataka
\item Sistem upisuje datu transakciju
\item Sistem prikazuje ažuriran finansijski izveštaj \footnote{Koraci (d), (e), (f), (g), (h), (i), (j) ponavljaju se sve dok ima transakcija koje je potrebno zavesti u izveštaj}
\item Direktor pritiska dugme za zaključivanje izveštaja
\item Sistem čuva podatke
\item Sistem vraća poruku o uspešno kreiranom izveštaju

\end{enumerate}

6. \textbf{Alternativni tokovi:} 
\begin{enumerate} [label=(\roman*)]
\item Direktor unosi nevalidne podatke u formular - Ako u koraku (h) sistem utvrdi neispravno polje formulara sistem obaveštava nastavnika obeležavanjem nevalidnog polja. Direktor ispravlja neispravno polje. Proces se nastavlja iz koraka (f).
\end{enumerate}

7. \textbf{Podtokovi}:  - \\

8. \textbf{Specijalni zahtevi}: - \\

9. \textbf{Dodatne informacije}: Obavezna polja formulara jesu: datum, tip transakcije (priliv/odliv), iznos. Opciono polje je napomena. Ovaj slučaj upotrebe služi samo za dokumentovanje transakcija, ne i za njihovo obavljanje. Finansijski izveštaj pravi se na mesečnom nivou.  \\


\subsubsection{Slučaj upotrebe: Direktor izriče učeniku težu disciplinsku meru}
1. \textbf{Kratak opis:} Na osnovu donesene odluke direktor unosi u sistem težu disciplinsku meru izrečenu učeniku. Sistem validira i čuva unete podatke i o tome obaveštava direktora. \\

2. \textbf{Učesnici:}
\begin{enumerate} [label=(\alph*)]
\item Direktor
\end{enumerate} 

3. \textbf{Preduslov:} Direktor je registrovani korisnik sistema. Direktor ima pristup Internetu. Sistem je u funkciji. \\

4. \textbf{Postuslov:} Dokument o izrečenoj meri je unet u arhivu dokumenata sistema. Disciplinska mera učenika evidentirana je u sistemu. Baza je ažurirana. \\

5. \textbf{Osnovni tok:} 
\begin{enumerate} [label=(\alph*)]
\item Direktor pristupa stranici sa listom svih učenika podeljenih po razredima i odeljenjima 
\item Direktor pritiska na dugme za pristup korisničkom profilu odgovarajućeg učenika
\item Sistem otvara profil učenika
\item Direktor pritiska dugme za kreiranje dokumenta o disciplinskoj meri
\item Sistem prikazuje obrazac za kreiranje dokumenta o disciplinskoj meri 
\item Direktor unosi tražene podatke
\item Direktor pritiskom na dugme potvrđuje unos
\item Sistem vrši validaciju podataka
\item Sistem čuva podatke
\item Sistem obaveštava direktora o uspešno izrečenoj disciplinskoj meri
\end{enumerate}

6. \textbf{Alternativni tokovi:}
\begin{enumerate} [label=(\roman*)]
\item Direktor unosi nevalidne podatke u obrazac - Ako u koraku (h) sistem utvrdi nevalidno polje obrasca (prazno obavezno polje) sistem obaveštava direktora obeležavanjem neispravnog polja. Direktor ispravlja neispravno polje. Proces se nastavlja iz koraka (f).
\end{enumerate}

7. \textbf{Podtokovi}: - \\

8. \textbf{Specijalni zahtevi}: Roditelj ili staratelj upoznat je sa disciplinskim postupkom koji se vodi protiv učenika. U slučaju ukora nastavničkog veća ili isključenja iz škole prethodno je održana sednica nastavičkog veća na kojoj je disciplinska mere izglasana. \\

9. \textbf{Dodatne informacije}: Obavezna polja obrasca jesu: datum, nivo mere (ukor direktora/ukor nastavničkog veća/isključenje učenika iz škole), razlog izricanja mere. Opciono polje je napomena. Lakše disciplinske mere povlače smanjenje ocene iz vladanja (ukor direktora na 4 ili 3, ukor nastavničkog veća na 2, isklučenje iz škola na 1). Nastavničko veće čine svi nastavnici zaposleni u školi. \\


\newpage
\subsection{Aktivnosti Administratora}




\end{document}