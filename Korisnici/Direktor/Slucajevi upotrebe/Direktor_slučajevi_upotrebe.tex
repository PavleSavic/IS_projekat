\documentclass{article}
\usepackage[utf8]{inputenc}
\usepackage{enumitem}
\usepackage{cmsrb}
\usepackage[OT2,T1]{fontenc} 
\usepackage[serbian]{babel}


\title{Nastavnici - slučajevi upotrebe}


\begin{document}

\maketitle

\underline{\textbf{Slučaj upotrebe:} Upisivanje novog nastavnika} \\

1. \textbf{Kratak opis:} Direktor dodaje novog nastavnika u spisak registrovanih korisnika sistema.\\

2. \textbf{Učesnici:}
\begin{enumerate} 
\item Direktor, nastavnik
\end{enumerate} 

3. \textbf{Preduslov:} Direktor je registrovani korisnik sistema. Direktor ima pristup Internetu. Sistem je u funkciji. Nastavnik koji se upisuje nije postojeći korsnik sistema.\\

4. \textbf{Postuslov:}  Novi nastavnik je registrovani korisnik sistema i dodeljeni su mu određeni časovi. \\

5. \textbf{Osnovni tok:} 
\begin{enumerate} 
\item Direktor klikće na dugme " {Dodaj} novog nastavnika"
\item Sistem prikazuje formu za upis novih nastavnika
\item Direktor popunjava formu odgovarajućim informacijama o novom nastavniku
\item Direktor klikće na dugme " {Dodaj}  nastavnika"
\item Sistem vrši proveru da li korisnik već postoji u sistemu
\item Sistem vraća poruku o uspešnom registrovanju novog nastavnika
\item Sistem prikazuje informacije o novom nastavniku
\item Direktor klikće na dugme " {Dodaj}  časove"
\item Direktor unosi nazive odeljenja i časove kojima novi nastavnik drži nastavu
\item Sistem vrši proveru da li za data odeljenja i časove ima nastavnika koji im drži nastavu
\item Sistem vraća poruku o uspešnom dodeljivanju časova novom nastavniku
\end{enumerate}

6. \textbf{Alternativni tokovi:}
\begin{enumerate} 
\item Korisnik već postoji u sistemu - Ako sistem u koraku (5) utvrdi da nastavnik sa postojećim informacijama već postoji u sistemu, sistem ispisuje odgovarajuću poruku i vraća korisnika na korak (2) 
\item Dodeljene časove već drže drugi nastavnici - Ako sistem u koraku (10) utvrdi da neke od dodeljenih časova novom nastavniku već drže drugi nastavnici, sistem ispisuje odgovarajuću poruku i vraća korisnika na korak (8)
\end{enumerate}

7. \textbf{Podtokovi}: - \\

8. \textbf{Specijalni zahtevi}: - \\

9. \textbf{Dodatne informacije}: Novi nastavnik mora imati završen fakultet iz određene oblasti koju želi da predaje učenicima.
Novom nastavniku ne moraju biti dodeljeni časovi koje drži. U formi u koraku (3) upisuju se predmeti koje nastavnik može da predaje i čuvaju se u sistemu.\\


\underline{\textbf{Slučaj upotrebe:} Brisanje postojećeg nastavnika} \\

1. \textbf{Kratak opis:} Sistem direktoru prikazuje trenutno zaposlene nastavnike u školi. Direktor briše postojećeg nastavnika iz spisak registrovanih korisnika sistema.\\

2. \textbf{Učesnici:}
\begin{enumerate} 
\item Direktor, nastavnik
\end{enumerate} 

3. \textbf{Preduslov:} Direktor je registrovani korisnik sistema. Direktor ima pristup Internetu. Sistem je u funkciji. Nastavnik koji se briše je postojeći korsnik sistema.\\

4. \textbf{Postuslov:}  Nastavnik je izbačen sa spiska registrovanih korisnika i onemogućen mu je pristup sistemu. \\

5. \textbf{Osnovni tok:} 
\begin{enumerate} 
\item Direktor klikće na dugme " Izbriši nastavnika"
\item Sistem prikazuje direktoru spisak zaposlenih nastavnika
\item Direktor bira nastavnika kojeg želi da izbriše iz sistema
\item Sistem prikazuje formu za ispisivanje nastavnika
\item Direktor popunjava formu odgovarajućim informacijama nastavniku
\item Direktor klikće na dugme " Izbriši  nastavnika"
\item Sistem briše nastavnika iz spiska registrovanih korisnika
\item Sistem za svako odeljenje kojima je nastavnik predavao određeni čas označava da trenutno nema dodeljenih profesora za držanje tog časa
\item Sistem vraća poruku o uspešnom brisanju nastavnika sa spiska registrovanih korisnika
\item Sistem prikazuje formu za dodeljivanje nastavnika odeljenjima kojima je izbrisani nastavnik predavao
\item Direktor određuje nastavnike koji drže časove odeljenjima kojima je izbrisani nastavnik predavao
\item Direktor klikće na dugme " {Dodeli} nastavnike"
\item Sistem dodeljuje časove odabranim nastavnicima
\item Sistem šalje poruku o uspešnoj dodeli nastavnika odeljenjima kojima je izbrisani nastavnik predavao 
\end{enumerate}

6. \textbf{Alternativni tokovi:}
\begin{enumerate} 
\item Nastavnik ne može da drži nastavu dodeljenim odeljenjima - Ukoliko dodeljeni nastavnik u koraku (13) ne može da predaje dodeljeni predmet, Sistem obaveštava klijenta i vraća ga na korak (10).
\end{enumerate}

7. \textbf{Podtokovi}: - \\

8. \textbf{Specijalni zahtevi}: - \\

9. \textbf{Dodatne informacije}: - \\


\underline{\textbf{Slučaj upotrebe:} Pristup kalendaru aktivnosti} \\

1. \textbf{Kratak opis:} Sistem direktoru prikazuje kalendar aktivnosti. Direktor može praviti izmene u kalendaru.\\

2. \textbf{Učesnici:}
\begin{enumerate} 
\item Direktor
\end{enumerate} 

3. \textbf{Preduslov:} Direktor je registrovani korisnik sistema. Direktor ima pristup Internetu. Sistem je u funkciji.\\

4. \textbf{Postuslov:}  Kalendar aktivnosti je ažuriran novim informacijama.\\

5. \textbf{Osnovni tok:} 
\begin{enumerate} 
\item Direktor klikće na dugme " Prikaži kalendar aktivnosti" 
\item Sistem prikazuje direktoru kalendar aktivnosti
\item Direktor klikom na dugme "Izmeni kalendar aktivnosi" pralazi na podtok slučaja upotrebe (1) 
\end{enumerate}

6. \textbf{Alternativni tokovi:}
\begin{enumerate} 
\item Nedovoljan broj radnih dana - Ukoliko u koraku (6) u podtoku (1) nakon izmene broj radnih dana ne ispunjava zakonski minimum,  Sistem šalje obaveštenje direktoru i onemogućava se klik na dugme "Završi" i prelazi se na korak (1).
\end{enumerate}

7. \textbf{Podtokovi}:  \\
Podtok 1:
\begin{enumerate}
\item Sistem direktoru prikazuje novi formular sa opcijama " {Dodaj nastavni dan} ", " {Dodaj vannastavnu aktivnost} " i "{ Dodaj} nenastavni dan"
\item Direktor klikće na odgovarajuće dugme
\item Sistem prikazuje direktoru formu u kojoj određuje datumski interval važenja izmene
\item Direktor unosi datum u formular
\item Sistem unosi izmenu u kalendar aktivnosti
\item Sistem šalje poruku o uspešnoj izmeni kalendara aktivnosti
\item Direktor klikće na dugme " {Završi} " 
\item Sistem vraća direktora na korak (2) iz osnovnog toka 
\end{enumerate} 

8. \textbf{Specijalni zahtevi}: - \\

9. \textbf{Dodatne informacije}: - \\


\underline{\textbf{Slučaj upotrebe:} Upisivanje novog učenika} \\

1. \textbf{Kratak opis:}  Direktor upisuje novog učenika u školu i dodeljuje mu odeljenje. \\

2. \textbf{Učesnici:}
\begin{enumerate} 
\item Direktor, učenik
\end{enumerate} 

3. \textbf{Preduslov:} Direktor je registrovani korisnik sistema. Direktor ima pristup Internetu. Sistem je u funkciji. Učenik nije upisan u drugu školu\\

4. \textbf{Postuslov:}  Učenik je registrovani korisnik sistema.\\

5. \textbf{Osnovni tok:} 
\begin{enumerate} 
\item Direktor klikće na dugme " {Dodaj} novog učenika " 
\item Sistem prikazuje formu za upis novih učenika
\item Direktor popunjava formu odgovarajućim informacijama o novom učeniku
\item Sistem vrši proveru da li učenik već postoji u sistemu
\item Sistem vraća poruku o uspešnom registrovanju novog učenika
\item Sistem prikazuje informacije o novom učeniku
\item Direktor klikće na dugme " {Dodaj}  odeljenje"
\item Direktor unosi naziv odeljenja 
\item Sistem proverava da li dodeljeno odeljenje postoji
\item Sistem dodeljuje odeljenje novom učeniku 
\item Sistem vraća poruku o uspešnom dodeljivanju odeljenja novom učeniku
\end{enumerate}

6. \textbf{Alternativni tokovi:}
\begin{enumerate} 
\item Učenik već postoji u sistemu - Ako sistem u koraku (4) utvrdi da učenik sa postojećim informacijama već postoji u sistemu, sistem ispisuje odgovarajuću poruku i vraća korisnika na korak (2) 
\item Dodeljeno odeljenje ne postoji - Ako sistem u koraku (9) utvrdi da dodeljeno odeljenje ne postoji, prikazuje odgovarajuću poruku i vraća korisnika na korak (8).
\end{enumerate}

7. \textbf{Podtokovi}:  - \\

8. \textbf{Specijalni zahtevi}: - \\

9. \textbf{Dodatne informacije}: - Tokom upisa novog učenika, i učenik i bar jedan roditelj ili staratelj moraju biti prisutni, roditelji moraju poneti ličnu kartu sa sobom i izvod iz matične knjige rođenih za učenika koji se upisuje.\\


\underline{\textbf{Slučaj upotrebe:} Pristup dnevnicima} \\

1. \textbf{Kratak opis:}  Sistem prikazuje direktoru sve dnevnike. Direktor može vršiti izmene u dnevnicima. \\

2. \textbf{Učesnici:}
\begin{enumerate} 
\item Direktor
\end{enumerate} 

3. \textbf{Preduslov:} Direktor je registrovani korisnik sistema. Direktor ima pristup Internetu. Sistem je u funkciji. \\

4. \textbf{Postuslov:}  Dnevnici su ažurirani ukoliko su izvršene izmene.\\

5. \textbf{Osnovni tok:} 
\begin{enumerate} 
\item Direktor klikće na dugme " {Pogledaj dnevnike}" 
\item Sistem prikazuje sve postojeće dnevnike
\item Direktor odabira dnevnik željenog odeljenja
\item Sistem pokazuje direktoru informacije sadržane u dnevniku
\item Ukoliko direktor klikne na dugme " {Izmena} ocena"  {sistem} prebacuje izvršavanje na podtok slučaja upotrebe (1)
\item Ukoliko direktor klikne na dugme " {Dodeljivanje} razrednog starešine" sistem prebacuje izvršavanje na podtok slučaja upotrebe (2)
\item Ukoliko direktor klikne na dugme " {Pravdanje izostanka}" {sistem} prebacuje izvršavanje na podtok slučaja upotrebe (3)
\item Ukoliko direktor klikne na dugme " {Dodeli zamenu}"  {sistem} prebacuje izvršavanje na podtok slučaja upotrebe (4)
\item Direktor klikće na dugme "Nazad" 
\item Sistem vraća direktora na početni ekran

\end{enumerate}

6. \textbf{Alternativni tokovi:}
\begin{enumerate} 
\item Učenik već postoji u sistemu - Ako sistem u koraku (4) utvrdi da učenik sa postojećim informacijama već postoji u sistemu, sistem ispisuje odgovarajuću poruku i vraća korisnika na korak (2) 
\item Dodeljeno odeljenje ne postoji - Ako sistem u koraku (9) utvrdi da dodeljeno odeljenje ne postoji, prikazuje odgovarajuću poruku i vraća korisnika na korak (8).
\end{enumerate}

7. \textbf{Podtokovi}:  \\
1: \\
\begin{enumerate}
\item Sistem prikazuje korisniku spisak učenika u odeljenju
\item Korisnik klikće na učenika
\item Sistem prikazuje sve ocene željenog učenika
\item Korisnik menja određenu ocenu
\item Korisnik klikće na  " {Završi} sa izmenama"
\item Sistem unosi izmene
\item Sistem vraća korisnika na korak (4) iz osnovnog toka
\end{enumerate}
2: \\
\begin{enumerate}
\item Sistem prikazuje korisniku spisak nastavnika
\item Korisnik klikće na nastavnika
\item Korisnik klikće na  " {Završi} sa izmenama"
\item Sistem unosi izmene
\item Sistem vraća korisnika na korak (4) iz osnovnog toka
\end{enumerate}
3: \\
\begin{enumerate}
\item Sistem prikazuje korisniku spisak učenika u odeljenju
\item Korisnik klikće na učenika
\item Sistem prikazuje sve izostanke željenog učenika
\item Korisnik menja određen izostanak
\item Korisnik klikće na  " {Završi} sa izmenama"
\item Sistem unosi izmene
\item Sistem vraća korisnika na korak (4) iz osnovnog toka
\end{enumerate}
4: \\
\begin{enumerate}
\item Sistem prikazuje korisniku spisak nastavnika koji predaju datom odeljenju
\item Korisnik klikće na nastavnika koji predaje tom odeljenju
\item Sistem prikazuje korisniku spisak nastavnika zaposlenih u školi
\item Korisnik klikće na nastavnika zaposlenog u školi
\item Sistem prikazuje korisniku sve predmete koje je nastavnik predavao odeljenju
\item Sistem prikazuje korisniku spisak nastavnika koji predaju datom odeljenju
\item Korisnik odabira predmete za koje želi da dodeli zamenu
\item Korisnik klikće na  " {Završi} sa izmenama"
\item Sistem unosi izmene
\item Sistem vraća korisnika na korak (4) iz osnovnog toka
\end{enumerate}

8. \textbf{Specijalni zahtevi}: - \\

9. \textbf{Dodatne informacije}: - \\


\underline{\textbf{Slučaj upotrebe:} Prikaz finansija} \\

1. \textbf{Kratak opis:}  Direktor ima uvid u finansijske dobitke i izdatke škole. \\

2. \textbf{Učesnici:}
\begin{enumerate} 
\item Direktor
\end{enumerate} 

3. \textbf{Preduslov:} Direktor je registrovani korisnik sistema. Direktor ima pristup Internetu. Sistem je u funkciji. \\

4. \textbf{Postuslov:}  - \\

5. \textbf{Osnovni tok:} 
\begin{enumerate} 
\item Direktor klikće na dugme " {Pogledaj finansije} " 
\item Sistem prikazuje tabelu svih finansijskih transakcija
\item Direktor klikće na dugme "{Nova transakcija}"
\item Sistem prikazuje formu za unošenje informacija o finansijskim transakcijama
\item Direktor unosi informacije o transakciji
\item Sistem upisuje datu transakciju
\item Sistem prikazuje ažuriranu zabelu svih transakcija
\end{enumerate}

6. \textbf{Alternativni tokovi:} -\\

7. \textbf{Podtokovi}:  - \\

8. \textbf{Specijalni zahtevi}: - \\

9. \textbf{Dodatne informacije}: - Ovaj slučaj upotrebe služi samo za upisivanje informacija o transakcijama i njihovom prikazivanju, ne i za obavljanje transakcija. \\

\end{document}