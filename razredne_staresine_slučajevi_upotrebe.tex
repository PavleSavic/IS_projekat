\documentclass{article}
\usepackage[utf8]{inputenc}
\usepackage{enumitem}
\usepackage{cmsrb}
\usepackage[OT2,T1]{fontenc} 
\usepackage[serbian]{babel}


\title{Razredne starešine - slučajevi upotrebe}


\begin{document}

\maketitle

\underline{\textbf{Slučaj upotrebe:} Razredni starešina opravdava učeniku evidentirani izostanak} \\

1. \textbf{Kratak opis:} Na osnovu opravdanja koje je učenik dostavio uživo ili elektronski skenirano kroz sistem razredni starešina evidentira izostanak kao opravdan. Sistem čuva promenu statusa odgovarajućeg izostanka i obaveštava razrednog starešinu. \\

2. \textbf{Učesnici:}
\begin{enumerate} [label=(\alph*)]
\item Razredni starešina
\item Učenik
\end{enumerate} 

3. \textbf{Preduslov:} Razredni starešina je registrovani korisnik sistema. Razredni starešina ima pristup Internetu. Sistem je u funkciji. Učenik je dostavio opravdanje. \\

4. \textbf{Postuslov:} Sistem je evidentirao izostanak učenika kao opravdan.\\

5. \textbf{Osnovni tok:} 
\begin{enumerate} [label=(\alph*)]
\item Razredni starešina pristupa formi i unosi kredencijale 
\item Sistem vrši autentifikaciju korisnika i odobrava pristup razrednom starešini
\item Razredni starešina pristupa stranici sa listom učenika u svom odeljenju
\item Razredni starešina pritiska na dugme za pristup korisničkom profilu odgovarajućeg učenika
\item Sistem otvara profil učenika
\item Razredni starešina pritiska na dugme za izlistavanje izostanaka učenika
\item Sistem prikazuje listu izostanaka i status svakog izostanka (opravdan/neopravdan)
\item Razredni starešina pritiska odgovarajući neopravdani izostanak
\item Sistem prikazuje dialog box (yes/no) i pita razrednog starešinu da li želi da promeni status neopravdanog izostanka
\item Razredni starešina potvrđuje promenu
\item Sistem čuva promenu statusa izostanka
\item Sistem obaveštava razrednog starešinu o uspešnoj promeni statusa izostanka
\end{enumerate}

6. \textbf{Alternativni tokovi:}
\begin{enumerate} [label=(\roman*)]
\item Razredni starešina unosi pogrešne kredencijale - Ako u koraku (b) sistem utvrdi pogrešne kredencijale (nepostojeće korisničko ime ili pogrešna lozinka) obaveštava razrednog starešinu obeležavanjem neispravnog polja. Razredni starešina ispravlja neispravno polje. Proces se nastavlja iz koraka (a).
\item Razredni starešina ne potvrđuje promenu statusa izostanka - Ako u koraku (j) razredni starešina odgovori odrično ili ugasi dialog box sistem ne evidentira promenu statusa. Proces se nastavlja iz koraka (g). 
\end{enumerate}

7. \textbf{Podtokovi}: - \\

8. \textbf{Specijalni zahtevi}: - \\

9. \textbf{Dodatne informacije}: Kredencijali razrednog starešine su korisničko ime i šifra. Dve nedelje nakon evidentiranja izostanka razredni starešina više nema mogućnost opravdavanja i status izostanka ostaje - neopravdan.\\



\underline{\textbf{Slučaj upotrebe:} Razredni starešina donosi odluku o lakšoj disciplinskoj meri za učenika} \\










\end{document}